\documentclass[12pt]{amsart}

%-------Packages---------
\usepackage{amssymb,amsfonts}
\usepackage{amsthm}
\usepackage{graphicx}
\usepackage[all,arc]{xy}
\usepackage[colorlinks=true, allcolors=blue]{hyperref}
\usepackage{enumerate}
\usepackage{bbm}
\usepackage{dsfont}
\usepackage{mathrsfs}
\usepackage{tikz-cd}
\usepackage{import}
\usepackage[utf8]{inputenc}
\usepackage{hyperref}
\usepackage{tikz}

\usepackage[left=1in,top=1in,right=1in,bottom=1in]{geometry}

\usepackage{mathtools}

%--------Theorem Environments--------
%theoremstyle{plain} --- default
\newtheorem{thm}{Theorem}[section]
\newtheorem{cor}[thm]{Corollary}
\newtheorem{prop}[thm]{Proposition}
\newtheorem{lem}[thm]{Lemma}
\newtheorem{conj}[thm]{Conjecture}

\theoremstyle{definition}
\newtheorem{defn}[thm]{Definition}
\newtheorem{defns}[thm]{Definitions}
\newtheorem{con}[thm]{Construction}
\newtheorem{example}[thm]{Example}
\newtheorem{exmps}[thm]{Examples}
\newtheorem{notn}[thm]{Notation}
\newtheorem{notns}[thm]{Notations}
\newtheorem{addm}[thm]{Addendum}
\newtheorem{exer}[thm]{Exercise}
\newtheorem{rmk}[thm]{Remark}
\newtheorem{fct}[thm]{Fact}
\newtheorem{quest}[thm]{Question}
\newtheorem{obsv}[thm]{Observation}

%----------- my commands -------------


%%%%%%%%% mathbb %%%%%%%%

\def\N{\mathbb{N}}
\def\Z{\mathbb{Z}}
\def\Q{\mathbb{Q}}
\def\R{\mathbb{R}}
\def\C{\mathbb{C}}


% comments
\newcommand{\cy}[1]{\textcolor{red}{#1}}

% Meta data

\title{Schur polynomials}
\author{Claudia He Yun}
\address{MPI MiS, 04103 Leipzig, Germany}
\email{\url{clyun@mis.mpg.de}}

\begin{document}

%\begin{abstract}
%\end{abstract}
	
\maketitle

\section{Question}
The irreducible representations of the special unitary group $SU(3)$ are indexed by Young diagrams with exactly two rows, i.e., partitions $\lambda$ with two parts. Now consider Schur polynomials in three variables $x_1,x_2,x_3$ obtained from these Young diagrams. We obtain the monomials in $s_\lambda$ by enumerating semistandard Young tableaux of shape $\lambda$.

\begin{example}
We have \[s_{2,1}(x_1,x_2,x_3) = x_1^2x_2+x_1^2x_3 + x_1x_2^2 + 2x_1x_2x_3 + x_1x_3^2 + x_2^2x_3 + x_2x_3^2\]
\end{example}

Let $P = \sum c_\lambda s_\lambda$, $c_\lambda > 0$, be a positive linear combination of Schur polynomials. Given a vector $v=(a,b,c)$ such that $a+b+c=0$, we may perform the substitution $P_v(z) = P(z^a,z^b,z^c)$.

\begin{example}
Let $P = s_{2,1}$. If $v = (1,0,-1)$, then $P_v(z) = z^2 + 2z +2+ 2z^{-1} + z^{-2}$. If $w = (1,2,-3)$, then $P_w(z) = z^5+z^4+z +2+z^{-1} + z^{-4} + z^{-5}$.
\end{example}

\begin{quest}
Can we find $P$ and vectors $v\neq w \neq 0$ such that $P_v(z) = P_w(z)$?
\end{quest}

\begin{obsv}
When $P = s_{2,1}$, we have $P_v(z) = P_{-v}(z)$. However, this is not true in general. For example, $s_{2,2}(1,2,-3)(z) = z^6 + z^2 + z + z^{-2} + z^{-3} + z^{-4}$ but $s_{2,2}(-1,-2,3)(z) = z^4 + z^3 + z^2 + z^{-1} + z^{-2} + z^{-6}$.
\end{obsv}

\begin{lem}
In $P_{(a,b,c)}(z)$, the sum of the exponents of $z$ is zero.
\end{lem}

\begin{proof}
The statement reduces to the case where $P$ is a single Schur polynomial. The sum of the exponents of $x_1$ in $P$ must be the same as that of $x_2$, which must be the same as that of $x_3$, because otherwise $P$ is not symmetric. Let that number be $k$. In $P_v(z)$, the sum of exponents of $z$ is $(a+b+c)k$, but $a+b+c=0$. 
\end{proof}

\bibliographystyle{alpha}
\bibliography{reference}

\end{document}